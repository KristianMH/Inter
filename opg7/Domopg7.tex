;; This buffer is for notes you don't want to save, and for Lisp evaluation.
;; If you want to create a file, visit that file with C-x C-f,
;; then enter the text in that file's own buffer.

forstyrende elementer paa siden saa som det der oejet men ikke ved hvad den goer.
Det er underligt at man skal oprettet en gruppe allerede naar man opreter et konto.
Naar man klikker paa x spoerg den ikke om man vil slette beskeden, den slettes bare hvilke man kan komme til at goere uden at vide det.
De fleste synes at roed var farven der skulle vaelges naar opgaven var vigtig.
Det er meget godt det kan tilsluttes med google Calender.

Intro kunne goeres bedre med mindre tekst, vise hvor de vigtige funktioner er, saasom oprette en ny gruppe, oploade en fil, tilfoele et nyt
medlem osv.

Det kunne have vaeret smart at der kom et diskret popup vindue naar man flytter musen for eksemple til opret ny gruppe ikon.

Det var svaert for nogen testdeltager at have nogen forventninger der de ikke havde nogen andelse om hvad gruppe rum er for noget.
men de fleste af dem blev ret hurtige klar over hvad det var efter de var kommet ind paa grouproom webside.

Vi fandt at testdeltager var hel overordnet genarelt positiv over for hjemmesiden. Selv om de synes der var nogen ting kunne v\ae re anderledes 
her og der. 

\Testdeltagernes oplevelse af webstedet

Mange testdeltager var begejstret for nogen funktionner. Vi kan for eksemple navne opload file funktionen som de fleste synes var godt 
gennemf\o rt i det den tilb\o d mange valg muligheder og den havde ogs\aa  ikoner der gjorde det nermmere at vaelge hvor man vil hentet sin file fra, om det s\aa  var fra dropbox eller computer, var der nemlig dropbox-ikonnet og computers ikonnet. 
Nogen andre deltager synes Task funtionen var smart lavet fordi det var mulig at rykket et eventuelt opgave mellem prioriteterne. For eks hvis
en opgave der havde mindre prioritet til at starte med kunne den for eksemple rykkes til h\o j prioritet hvis dens status \ae ndrede sig efter 
p\aa  et senere tidspunkt. P\aa  den m\aa de beh\o ver man ikke at skulle opretet opgaven p\aa  ny for at \ae ndre dens prioritet.
Test deltagere syntes ogs\aa  m\aa den man sletter folk p\aa  var snart i det der dukker et popup vindue som sp\org om man er sikker man vil 
slette den p\aa g\aeldende gruppemedlem fra gruppe, man har derfor muligheden til at bekr\ae fte eller afkr\aefte det. 

Det skal skal altsaa sige at nettop den lille meddelse er der nogen der savnede naar man skal slette beskeder. de
mente memlig at man kunne komme til at slette beskeder uden at det var mening da det ikke er nogen advaverlsels system. 

Vi testede hjemsiden, hvor en medlem af vores gruppe testede testopgavene for at se om de kunne udf\o res og eventuelt revurderer dem fandt vi
ud af at nogen opgaverne var sv\ae rt at udf\o re, Dermed var det forventet at de vil give problemer for testdeltager. det var ikke fordi 
opgaverne var d\aa rlige formuleret fordi deltager forst\aa godt hvad de var blevet bedt om, men det mere fordi l\o sningen ikke bare 
sprang ud af hjemsiden.
Vi har for eksemple opret en ny gruppe, den var sv\ae rt at l\o se b\aa de for vores gruppemedlem men ogs\aa  for de flertal af deltagerne.
Det skyldes m\aa ske den valgte ikonen, som er godt gemt i hj\o re hj\o rnet. De fleste deltager havde den samme refleks hvilke hvergang de 
skulle loese en opgave, det var at bev\ae ge musen paa de forskellige ikonner p\aa hjemmesiden. Det skete mange gang at de bev\ae gede mussen 
forbi den ikonnen de skulle klikke sig ind i for at l\o se opgaven.      