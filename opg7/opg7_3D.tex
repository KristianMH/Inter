\documentclass[12pt]{article}
\usepackage[a4paper, hmargin={2.8cm, 2.8cm}, vmargin={2.5cm, 2.5cm}]{geometry}
\usepackage{eso-pic} % \AddToShipoutPicture

\usepackage[utf8]{inputenc}
\usepackage[T1]{fontenc}
\usepackage{lmodern}
\usepackage[english]{babel}
\usepackage{cite}
\usepackage{amssymb}
\usepackage{amsfonts}
\usepackage{amsmath}
\usepackage{enumerate}
\usepackage{mathrsfs}
\usepackage{fullpage}
\usepackage[linkcolor=red]{hyperref}
\usepackage[final]{graphicx}
\usepackage{color}
\usepackage{listings}
\renewcommand*\lstlistingname{Code Block}
\definecolor{bg}{rgb}{0.95,0.95,0.95}

%caption distinct from normal text
\usepackage[hang,small,bf]{caption}
\usepackage{hyperref}

\hypersetup{
    colorlinks,%
    citecolor=black,%
    filecolor=black,%
    linkcolor=black,%
    urlcolor=black
}

\author{
  \texttt{Gruppe: 3D} \\
  \texttt{Mikkel Enevoldsen} \\[.4cm]
  \texttt{Kristian Høi} \\[.4cm]
  \texttt{Dominique Chancelier} \\[.4cm]
  \texttt{Carsten Jensen} \\[.4cm]
  Instruktor: Jesper Lundsgaard
  \vspace{8cm}
}

\title{
  \vspace{3cm}
  \Huge{Opgave 6} \\[.25cm]
  \large{Kontekstuel Analyse}
  \vspace{.75cm}
}

\begin{document}

\AddToShipoutPicture*{\put(0,0){\includegraphics*[viewport=0 0 700 600]{includes/ku-farve}}}
\AddToShipoutPicture*{\put(0,602){\includegraphics*[viewport=0 600 700 1600]{includes/ku-farve}}}

%% Change `ku-en` to `nat-en` to use the `Faculty of Science` header
\AddToShipoutPicture*{\put(0,0){\includegraphics*{includes/ku-en}}}

\clearpage\maketitle
\thispagestyle{empty}

\newpage

\tableofcontents %generate table of content

\thispagestyle{empty}

\newpage
\pagestyle{plain}
\setcounter{page}{1}
\pagenumbering{arabic}

\section{Resumé}

\section{Indledning}

\section{Testdeltagernes forventninger til webstedet}

\section{Testdeltagernes oplevelse af webstedet}

\section{Fremgangsmåde}

\section{Drejebog}
Før hver testsession skal computeren være tændt, browseren skal stå på ny fane siden i inkognito tilstand. For en sikkerhedsskyld skal historikken være slettet.
\subsection{Stikord før interview}
Vi tester ikke dig men web-stedet. \\
Vi må ikke hjælpe dig under testen. \\
Du skal tænke højt under testen. \\
Du kan ikke lave fejl
\subsection{Spørgsmål før interview}
\begin{enumerate}
  \item Hvad er dine websurfing/it erfaringer?
  \item Kender du Grouproom, hvis ja, har du brugt det før og hvad foretog du dig?
  \item Kender du andre hjemmesider der minder om det?
\end{enumerate}
\subsection{Testopgaver}
\begin{enumerate}
\item Find Grouprooms hjemmeside.
\item Opret profil på Grouproom og login og udforsk hjemmesiden.
\item Har du en studiegruppe du studerer med? Hvis ja, så opret en studiegruppe til gruppen.
\item Tilføj dine studiekammerater til denne gruppe, og hvis de ikke er oprettet på GroupRoom, tilføj dummy-brugeren Mikkel.
\item Upload en fil til gruppen og rediger i denne fil.
\item Send en besked til gruppemedlemmerne om at vil slette filen igen.
\item Slet filen.
\item Opret en begivenhed i din gruppes kalender.
\item Slet begivenheden i din gruppes kalender.
\item Opret en opgave og uddelegér den til et andet medlem.
\item Ekskludér et medlem af gruppen.
\end{enumerate}
\subsection{Spørgsmål til eftersnak}
\begin{itemize}
  \item Hvordan synes du overordnet denne test har været?
  \item Hvad er de 3 bedste ting ved hjemmesiden?
  \item Hvad er de 3 værste ting ved hjemmesiden?
  \item Kunne du finde på at bruge denne hjemmeside igen?
\end{itemize}
\section{Testdeltagernes opgaveløsning}

\section{Inspektion af websted inden test}

\section{Sammenligning med inspektion}

\section{Kommentarer}


\end{document}
