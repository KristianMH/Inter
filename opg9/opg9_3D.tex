\documentclass[12pt]{article}
\usepackage[a4paper, hmargin={2.8cm, 2.8cm}, vmargin={2.5cm, 2.5cm}]{geometry}
\usepackage{eso-pic} % \AddToShipoutPicture

\usepackage[utf8]{inputenc}
\usepackage[T1]{fontenc}
\usepackage{lmodern}
\usepackage[english]{babel}
\usepackage{cite}
\usepackage{amssymb}
\usepackage{amsfonts}
\usepackage{amsmath}
\usepackage{enumerate}
\usepackage{mathrsfs}
\usepackage{fullpage}
\usepackage[linkcolor=red]{hyperref}
\usepackage[final]{graphicx}
\usepackage{color}
\usepackage{listings}
\renewcommand*\lstlistingname{Code Block}
\definecolor{bg}{rgb}{0.95,0.95,0.95}

%caption distinct from normal text
\usepackage[hang,small,bf]{caption}
\usepackage{hyperref}

\hypersetup{
    colorlinks,%
    citecolor=black,%
    filecolor=black,%
    linkcolor=black,%
    urlcolor=black
}

\author{
  \texttt{Gruppe: 3D} \\
  \texttt{Mikkel Enevoldsen} \\[.4cm]
  \texttt{Kristian Høi} \\[.4cm]
  \texttt{Dominique Chancelier} \\[.4cm]
  \texttt{Carsten Jensen} \\[.4cm]
  Instruktor: Jesper Lundsgaard
  \vspace{8cm}
}

\title{
  \vspace{3cm}
  \Huge{Opgave 9} \\[.25cm]
  \large{Sammenligning af usabilitytestresultater}
  \vspace{.75cm}
}

\begin{document}

\AddToShipoutPicture*{\put(0,0){\includegraphics*[viewport=0 0 700 600]{includes/ku-farve}}}
\AddToShipoutPicture*{\put(0,602){\includegraphics*[viewport=0 600 700 1600]{includes/ku-farve}}}

%% Change `ku-en` to `nat-en` to use the `Faculty of Science` header
\AddToShipoutPicture*{\put(0,0){\includegraphics*{includes/ku-en}}}

\clearpage\maketitle
\thispagestyle{empty}

\newpage

\thispagestyle{empty}

\newpage
\pagestyle{plain}
\setcounter{page}{1}
\pagenumbering{arabic}


\subsection*{Vores gruppes problemliste}
\begin{enumerate}
\item Fejlbedskeden ved "Get-started" signup er forvirrende. ved glemt "@" fortæller systemet bare at man skal rette sine informationer så det bliver grønne og ikke hvad der skal rettes.
\item Flere af vores brugere opdagede ikke "Thank you for signing up" popup ved tildmelding, derved vidste de ikke at de skulle validere deres email for at logge ind.
\item Mange af testdeltagerne fandt navigationen mellem Files/Gallery og Task/Calender
forvirrende - og flere endte op med at bruge andre alternativer til at løse opgaver,
der ellers havde til formål at teste disse.
\item Task bliver klart fremhævet, mens calendar må
indtage en sekundær position. Det gør det problematisk, da de fleste af vores
testdeltagere valgte fuldt ud at ignorere den, hvilket gør den ubrugelig
\item Søgebaren dukker op på mystisk vis. De fleste testdeltagere
kunne rigtig nok genkende det som en søgebar, men var meget forvirrede om hvad
de egentlig søgte på
\item "Take the tour" er overflødig, da 60\% af vores testdeltagere valgte at springe den over og de resterende 40\% fandt den ubrugelig og for lang.
\item Når testdeltagerne skulle oprette et nyt grupperum til sine medstuderende, var
der problemer med at finde knappen. Dette udløste lettere irritation fra flere, der
ikke kunne finde knappen
\item Nogle ikoner er ens, men betyder forskellige ting for forskellige områder. Det
forvirrede nogle brugere, da det forventedes at de ville have samme funktionalitet.
Et eksempel på dette, er når man vil uploade en
fil og tilføje et medlem - to funktioner med det samme ikon. Det brokkede nogle af
testdeltagerne sig over, da det gav anledning til flere fejlklik.
\item En bruger så det unødvendigt at dele files og gallery op i to faner, fulgt op med et
retorisk spørgsmål om hvorvidt billeder ikke også var filer. Han så kun fordele i at
samle de to til en, om ikke andet så for at undgå forvirring omkring forskellene på
dem.
\item Det lille flammeikon, der skal indikere når der er sket noget nyt på i en gruppe,
fandt flere mere forstyrrende end gavnlig.
\end{enumerate}

\subsection*{Fællesmøde problemliste}
\begin{enumerate}
  \item  Fejlmeddelser ved signunp/login
  \item  Forvirrende ikoner
  \item  Mystisk søgebar 
  \item  Ikke muligt at strække events over flere dage
  \item  Forvirrende layout af hjemmesiden
  \item  Svært at finde new document
  \item  Problem med tilføj/slet gruppemedlem 
  \item  Beskræftelse af gruppe 
  \item  Forvirrende brug af faner
  \item  Svært at finde new event
\end{enumerate}
\subsection*{Kommentarer}
\begin{enumerate}
  \item [Problem 1] Ligner problem 1 og 2, da dette problem både omfanger fejlmeddelser ved signup og email-verifikation.
  \item [Problem 2] Ligner vores problem 8, hvor enslignende ikoner havde forskellige funktioner. Derudover ligner det også problem 10 med uforklarligt flammeikon. 
  \item [Problem 3] Er identitisk med problem 5 fra vores liste.
  \item [Problem 4] Dette problem står ikke i vores rapport, vi bemærkede ikke dette problem, da vi ikke havde en opgave hvor man skulle oprette et event som foregik over flere dage.
  \item [Problem 5] har vi ikke med i vores rapport. Vi synes, at dette problem er forkert formuleret og alt for bredt til at blive beskrevet. Det ville være passende at opdele det i mindre problemer. Vores problem med files/gallery og task/calender fra rapporten ligner dog dette problem.
  \item [Problem 6] står ikke i vores rapport, Vi bemærkede ike dette problem, da vi ikke havde nogen opgaver med det, dog synes vi at det er et mindre problem.
  \item [Problem 7] identisk med et problem i vores rapport.
  \item [Problem 8] står ikke i vores rapport, men selvom vi ikke selv oplevede det i vores test, synes det som en relevant problem.
  \item [Problem 9] står ikke i vores rapport og da vi ikke har oplevet det for os selv eller vores testdeltagere, synes vi ikke problemet er relevant.
  \item [Problem 10] ligner et problem i vores rapport - vores deltagere kunne slet ikke finde kalenderen, og kunne derfor heller ikke oprette et nyt event.
   
\end{enumerate}
\subsection*{Konklusion}

Vi kan se at der var kun et problem der var identisk med et problem fra vores liste, problem 3 passer med problem 5 fra vores liste og nogen problemer der ligner noget fra vores liste for eksempel problem 1 som minder meget om vore problem 1 og 2. De andre problemer var lignede mere noget vi havde haft i vores rapport men, som vi ikke havde taget med i vores top ti liste og andre problemer havde vi simpelthen ikke i vores rapport enten fordi vores testdeltagere ikke havde haft dem som udfordring eller fordi vi ikke  havde opgaver, der kunne havde genereret disse. Det inkluderer eksempelvis problem 4 og 5. \\

\noindent Det er svært at sige om fællesmødets problemer var bedre end vores da vi havde forskelige vinkler og opgaver, hvilket gør det svært at sætte det op mod hindanden. Umiddelbart var der dog mange problemer grupperne imellem der lignede hinanden, som blev grupperet sammen til en bredere problematik. Vi synes KJ-metoden virkede udmærket til formålet, men måske ville det være det være mere optimalt, hvis alle havde de samme tests at forholde sig til.\\

\noindent Som det kan ses fra ovenstående kommentarer indeholder problemlisten fra fællesmødet både vores observationer og nye problemer. Størstedelen af de problem som vi ikke registrerede var mest fordi de havde et andet fokus i deres testopgaver og derved registrerede andre problem. Nogle af disse problemer mener vi dog er mindre vigtige.
\end{document}
