\documentclass[12pt]{article}
\usepackage[a4paper, hmargin={2.8cm, 2.8cm}, vmargin={2.5cm, 2.5cm}]{geometry}
\usepackage{eso-pic} % \AddToShipoutPicture

\usepackage[utf8]{inputenc}
\usepackage[T1]{fontenc}
\usepackage{lmodern}
\usepackage[english]{babel}
\usepackage{cite}
\usepackage{amssymb}
\usepackage{amsfonts}
\usepackage{amsmath}
\usepackage{enumerate}
\usepackage{mathrsfs}
\usepackage{fullpage}
\usepackage[linkcolor=red]{hyperref}
\usepackage[final]{graphicx}
\usepackage{color}
\usepackage{listings}
\renewcommand*\lstlistingname{Code Block}
\definecolor{bg}{rgb}{0.95,0.95,0.95}

%caption distinct from normal text
\usepackage[hang,small,bf]{caption}
\usepackage{hyperref}

\hypersetup{
    colorlinks,%
    citecolor=black,%
    filecolor=black,%
    linkcolor=black,%
    urlcolor=black
}

\author{
  \texttt{Gruppe: 3D} \\
  \texttt{Mikkel Enevoldsen} \\[.4cm]
  \texttt{Kristian Høi} \\[.4cm]
  \texttt{Dominique Chancelier} \\[.4cm]
  \texttt{Carsten Jensen} \\[.4cm]
  Instruktor: Jesper Lundsgaard
  \vspace{8cm}
}

\title{
  \vspace{3cm}
  \Huge{Opgave 6} \\[.25cm]
  \large{Kontekstuel Analyse}
  \vspace{.75cm}
}

\begin{document}

\AddToShipoutPicture*{\put(0,0){\includegraphics*[viewport=0 0 700 600]{includes/ku-farve}}}
\AddToShipoutPicture*{\put(0,602){\includegraphics*[viewport=0 600 700 1600]{includes/ku-farve}}}

%% Change `ku-en` to `nat-en` to use the `Faculty of Science` header
\AddToShipoutPicture*{\put(0,0){\includegraphics*{includes/ku-en}}}

\clearpage\maketitle
\thispagestyle{empty}

\newpage

\tableofcontents %generate table of content

\thispagestyle{empty}

\newpage
\pagestyle{plain}
\setcounter{page}{1}
\pagenumbering{arabic}

\section*{Resumé}


\section*{}

UPT:
- unique problem token
- eksempel fra vores egen: signup-problemet, det røde flammeikon etc.
- table 2 og figure 2 stemmer nogenlunde overens. Hvis man tager en tilfældig evaluator, vil de i alle tilfælde have omkring halvdelen UPT's hver. Hvis vi tager 2 tilfældige, vil de ligeledes også passe nogenlunde med de 66\% tabel 2 viser.
- Vi bruger formlen på side 2, og finder frem til 97,427. Det vil sige, at der hvis vi runder op til 98 er 5 UPT'er de ikke har fundet frem til.

Evaluator-effekt:
- Jo flere evaluatorer, des flere UPT'er finder man, men effekten synes at være stavnerende efterhånden som antallet af evaluatorer går op (qua figur 1). Det kræver naturligvis også flere ressourcer og mere tid, jo flere evaluatorer man skal have.
- Flere evaluatorer har også en tendens til at finde forskellige problemer, og de finder stor uenighed i deres prioritering af eksempelvis top 10-lister over største problemer. Det er naturligvis et problem.

Overraskelse:
Fire mennesker med markant viden, kunne ikke blive enige om hvad de mente var de mest kritiske problemområder.







\end{enumerate}
\end{document}
