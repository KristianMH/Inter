\documentclass[12pt]{article}
\usepackage[a4paper, hmargin={2.8cm, 2.8cm}, vmargin={2.5cm, 2.5cm}]{geometry}
\usepackage{eso-pic} % \AddToShipoutPicture

\usepackage[utf8]{inputenc}
\usepackage[T1]{fontenc}
\usepackage{lmodern}
\usepackage[english]{babel}
\usepackage{cite}
\usepackage{amssymb}
\usepackage{amsfonts}
\usepackage{amsmath}
\usepackage{enumerate}
\usepackage{mathrsfs}
\usepackage{fullpage}
\usepackage[linkcolor=red]{hyperref}
\usepackage[final]{graphicx}
\usepackage{color}
\usepackage{listings}
\renewcommand*\lstlistingname{Code Block}
\definecolor{bg}{rgb}{0.95,0.95,0.95}

%caption distinct from normal text
\usepackage[hang,small,bf]{caption}
\usepackage{hyperref}

\hypersetup{
    colorlinks,%
    citecolor=black,%
    filecolor=black,%
    linkcolor=black,%
    urlcolor=black
}

\author{
  \texttt{Gruppe: 3D} \\
  \texttt{Mikkel Enevoldsen} \\[.4cm]
  \texttt{Kristian Høi} \\[.4cm]
  \texttt{Dominique Chancelier} \\[.4cm]
  \texttt{Carsten Jensen} \\[.4cm]
  Instruktor: Jesper Lundsgaard
  \vspace{8cm}
}

\title{
  \vspace{3cm}
  \Huge{Opgave 4} \\[.25cm]
  \large{Målelige krav til brugsvenlighed}
  \vspace{.75cm}
}

\begin{document}

\AddToShipoutPicture*{\put(0,0){\includegraphics*[viewport=0 0 700 600]{includes/ku-farve}}}
\AddToShipoutPicture*{\put(0,602){\includegraphics*[viewport=0 600 700 1600]{includes/ku-farve}}}

%% Change `ku-en` to `nat-en` to use the `Faculty of Science` header
\AddToShipoutPicture*{\put(0,0){\includegraphics*{includes/ku-en}}}

\clearpage\maketitle
\thispagestyle{empty}

\newpage

%\tableofcontents %generate table of content

\thispagestyle{empty}

%\newpage
\pagestyle{plain}
\setcounter{page}{1}
\pagenumbering{arabic}

\section*{}
Oplæg: FIVA – Finde varer i et supermarked\\
\\
Problemformulering: Analysér behovet for en mobilapplikation, der kan hjælpe kunder med at finde varer, når de står i et
supermarked.\\
\\
Eksempel: Brugeren ved ikke, hvor han kan finde marcipan. Applikationen hjælper\\
ham med at finde den hylde, hvor marcipanen ligger.\\
\\
Formålene med applikationen er:\\
Kunderne skal bruge mindre tid på at finde varer.
Forretningens medarbejdere skal bruge mindre tid på at hjælpe kunder med at finde varer. Du kan gå ud fra, at mobiltelefonen ved præcis, hvor i butikken brugeren befinder sig. Applikationen skal baseres på eksisterende teknologi. Talegenkendelse er ikke en gyldig del af en løsning. Dette oplæg er bevidst formuleret vagt. Brug interviewene til at finde yderligere muligheder, som vil begejstre de kommende brugere af det planlagte produkt og sikre dets nytteværdi.

\section*{}

\textbf{introduction}

Hovedformålet med applikationen er, at forretningens medarbejdere skal bruge mindre tid på at hjælpe kunder med at finde varer.

"Vi er datalogistuderende fra Københanvs Universitet, som skal designe en applikation omhandlende at gøre det lettere for kunden at finde varer. Vi ønsker at bruge 10 minutter på at få dine tanker omkring en sådan applikation  i et interview."


\textbf{PACT:}

People
\begin{itemize}
\item Sprogforskelle hvilke sprog synes vil v\ae re passende for FIVA
\item Hukommelse 
\item Generthed (sociale udfordringer ved henvendelse om vareplacering)
\item Socialklasser (indkomstforskelle)
\item Indkøbserfaring
\end{itemize}

Activity
\begin{itemize}
\item Indkøbsfrekvens  
\item Tidspres
\item Formålet veldefineret: Handle ind.
\item Præventivt varetjek. 
\item   
\end{itemize}


Context
\begin{itemize}
\item Supermarkeder - indkøb 
\item  
\end{itemize}


Technology
\begin{itemize}
\item Smartphone -   
\item GPS-tilgængelighed
\item Hurtighed
\item Højtlæsning af resultater. 
\end{itemize}

\textbf{Tjekliste til interview:}\\
 
\begin{itemize}
\item Fakta (Alder, køn, teknologisk erfaring)\\
\item Socialklasse "Hvilket erhverv og/eller uddannelsesbaggrund har du?"\\
\item Indkøbsfrekvens "Hvor tit handler du ind - og i hvilket tidsrum?"\\
 \item Erfaring	"På hvilket niveau, erfaringsmæssigt, vil du beskrive dig selv som indkøber?"\\
\item Tidsfaktor "Hvor lang tid har du til rådighed, når du handler ind?"\\
\item Motivation "Hvilken tilgang har du til indkøb - har du eksempelvis en struktureret plan over varer, eller køber du hvad 			der falder dig ind?"\\
\item  Vareplacering "Fortæl mig om en situation, hvor du ikke har kunnet kunne finde en vare i et supermarked."\\
\item Hjælp "Hvor ofte må du spørge en medarbejder efter hjælp?"\\
\item Medarb.konfrontation "Hvilke udfordringer forbinder du med at skulle opsøge en medarbejder om en vares placering?"\\
\item Hukommelse "Hvor mange gange om måneden glemmer du hvad du skal købe i et supermarked? - Kan du fortælle om en 				specifik situation?"\\
\item Teknologisk vane "Hvis du har en smartphone, hvordan bruger du så den i forbindelse med indkøb?"\\
 \item Tekn. til/fravalg "Hvorfor foretrækker du smartphone frem for andre alternativer?"\\
\item Præventivt varetjek "Hvordan kunne det ændre din indkøbsrutine, hvis du hjemmefra kunne tjekke en vares placering?"
\item Forventning "Hvad ville du forvente en sådan applikation skulle indeholde?"\\
\item Hurtighed	"Hvor lang tid vil du sige, det højst burde tage at finde en vare ved hjælp af FIVA-appen, hvorfor?"\\
\item Nødvendighed "Ville du bruge en varelokaliserings-app til din smartphone, hvis sådan en fandtes - hvorfor?"\\
 \item Alternativer "Kan du komme på alternativer, der ville gøre en sådan applikation overflødig?"\\

x. Præventivt varetjek2	"Hvis vi kigger på de sidste 100 gange du har skullet ud at handle, hvor mange gange vil du så skyde på at du 		har ringer og spurgt i forvejen om de har en vare?"

\end{document}
