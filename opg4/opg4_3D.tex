\documentclass[12pt]{article}
\usepackage[a4paper, hmargin={2.8cm, 2.8cm}, vmargin={2.5cm, 2.5cm}]{geometry}
\usepackage{eso-pic} % \AddToShipoutPicture

\usepackage[utf8]{inputenc}
\usepackage[T1]{fontenc}
\usepackage{lmodern}
\usepackage[english]{babel}
\usepackage{cite}
\usepackage{amssymb}
\usepackage{amsfonts}
\usepackage{amsmath}
\usepackage{enumerate}
\usepackage{mathrsfs}
\usepackage{fullpage}
\usepackage[linkcolor=red]{hyperref}
\usepackage[final]{graphicx}
\usepackage{color}
\usepackage{listings}
\renewcommand*\lstlistingname{Code Block}
\definecolor{bg}{rgb}{0.95,0.95,0.95}

%caption distinct from normal text
\usepackage[hang,small,bf]{caption}
\usepackage{hyperref}

\hypersetup{
    colorlinks,%
    citecolor=black,%
    filecolor=black,%
    linkcolor=black,%
    urlcolor=black
}

\author{
  \texttt{Gruppe: 3D} \\
  \texttt{Mikkel Enevoldsen} \\[.4cm]
  \texttt{Kristian Høi} \\[.4cm]
  \texttt{Dominique Chancelier} \\[.4cm]
  \texttt{Carsten Jensen} \\[.4cm]
  Instruktor: Jesper Lundsgaard
  \vspace{8cm}
}

\title{
  \vspace{3cm}
  \Huge{Opgave 4} \\[.25cm]
  \large{Målelige krav til brugsvenlighed}
  \vspace{.75cm}
}

\begin{document}

\AddToShipoutPicture*{\put(0,0){\includegraphics*[viewport=0 0 700 600]{includes/ku-farve}}}
\AddToShipoutPicture*{\put(0,602){\includegraphics*[viewport=0 600 700 1600]{includes/ku-farve}}}

%% Change `ku-en` to `nat-en` to use the `Faculty of Science` header
\AddToShipoutPicture*{\put(0,0){\includegraphics*{includes/ku-en}}}

\clearpage\maketitle
\thispagestyle{empty}

\newpage

%\tableofcontents %generate table of content

\thispagestyle{empty}

%\newpage
\pagestyle{plain}
\setcounter{page}{1}
\pagenumbering{arabic}

\section*{}
Oplæg: FIVA – Finde varer i et supermarked\\
\\
Problemformulering: Analysér behovet for en mobilapplikation, der kan hjælpe kunder med at finde varer, når de står i et
supermarked.\\
\\
Eksempel: Brugeren ved ikke, hvor han kan finde marcipan. Applikationen hjælper\\
ham med at finde den hylde, hvor marcipanen ligger.\\
\\
Formålene med applikationen er:\\
Kunderne skal bruge mindre tid på at finde varer.
Forretningens medarbejdere skal bruge mindre tid på at hjælpe kunder med at finde varer. Du kan gå ud fra, at mobiltelefonen ved præcis, hvor i butikken brugeren befinder sig. Applikationen skal baseres på eksisterende teknologi. Talegenkendelse er ikke en gyldig del af en løsning. Dette oplæg er bevidst formuleret vagt. Brug interviewene til at finde yderligere muligheder, som vil begejstre de kommende brugere af det planlagte produkt og sikre dets nytteværdi.

\section*{}

Brainstorm\\
\\
kontekstuelt\\
Location\\
Neutralt sp\o rgsm\aa l\\
\\
Aldersgruppe - \\ 
K\o n\\
\\
App kendskab - \\
Smarphone ejerskab\\
Indsk\o bs frekvens\\

\textbf{PACT:}

People
\begin{itemize}
\item Sprogforskelle
\item Hukommelse
\item Generthed (sociale udfordringer ved henvendelse om vareplacering)
\item Socialklasser (indkomstforskelle)
\item Indkøbserfaring
\end{itemize}

Activity
\begin{itemize}
\item Indkøbsfrekvens
\item Tidspres
\item Formålet veldefineret: Handle ind.
\item Præventivt varetjek
\end{itemize}


Context
\begin{itemize}
\item Supermarkeder - indkøb
\end{itemize}


Technology
\begin{itemize}
\item Smartphone
\item GPS-tilgængelighed
\item Hurtighed
\item Højtlæsning af resultater.
\end{itemize}

\textbf{Tjekliste til interview:}
\begin{enumerate}
\item asdlæs
\item askdn
\item apsdma
\item klasnd
\item alsn
\end{enumerate}


Brainstorm

  kontekstuelt - Location \\

  Neutralt sp\o rgsm\aa l \\

  Aldersgruppe - K\o n \\

  App kendskab - Smarphone eller tablet ejerskab \\ 

  Indsk\o bs frekvens(hypighed) \\

  \AA bne sp\o rgsm\aa l \\ 

  Indk\o bsliste \\
  
  Indsk\o bs varighed \\
  
  Butiksk\ae de \\

  Vare \\
 
 En rettelse der skal virke.   


\end{document}
