\documentclass[12pt]{article}
\usepackage[a4paper, hmargin={2.8cm, 2.8cm}, vmargin={2.5cm, 2.5cm}]{geometry}
\usepackage{eso-pic} % \AddToShipoutPicture

\usepackage[utf8]{inputenc}
\usepackage[T1]{fontenc}
\usepackage{lmodern}
\usepackage[english]{babel}
\usepackage{cite}
\usepackage{amssymb}
\usepackage{amsfonts}
\usepackage{amsmath}
\usepackage{enumerate}
\usepackage{mathrsfs}
\usepackage{fullpage}
\usepackage[linkcolor=red]{hyperref}
\usepackage[final]{graphicx}
\usepackage{color}
\usepackage{listings}
\renewcommand*\lstlistingname{Code Block}
\definecolor{bg}{rgb}{0.95,0.95,0.95}

%caption distinct from normal text
\usepackage[hang,small,bf]{caption}
\usepackage{hyperref}

\hypersetup{
    colorlinks,%
    citecolor=black,%
    filecolor=black,%
    linkcolor=black,%
    urlcolor=black
}

\author{
  \texttt{Gruppe: 3D} \\
  \texttt{Mikkel Enevoldsen} \\[.4cm]
  \texttt{Kristian Høi} \\[.4cm]
  \texttt{Dominique Chancelier} \\[.4cm]
  \texttt{Carsten Jensen} \\[.4cm]
  Instruktor: Jesper Lundsgaard
  \vspace{8cm}
}

\title{
  \vspace{3cm}
  \Huge{Opgave 6} \\[.25cm]
  \large{Kontekstuel Analyse}
  \vspace{.75cm}
}

\begin{document}

\AddToShipoutPicture*{\put(0,0){\includegraphics*[viewport=0 0 700 600]{includes/ku-farve}}}
\AddToShipoutPicture*{\put(0,602){\includegraphics*[viewport=0 600 700 1600]{includes/ku-farve}}}

%% Change `ku-en` to `nat-en` to use the `Faculty of Science` header
\AddToShipoutPicture*{\put(0,0){\includegraphics*{includes/ku-en}}}

\clearpage\maketitle
\thispagestyle{empty}

\newpage

\tableofcontents %generate table of content

\thispagestyle{empty}

\newpage
\pagestyle{plain}
\setcounter{page}{1}
\pagenumbering{arabic}

\section*{underopgave 1}


Testleder: Carsten
Testdeltager: Dominique
Testdeltager "dummy": Mikkel1: Find Grouprooms hjemmeside.

\begin{itemize} 
\item {sp\o rgsm\aa l}

Du har f\aa et en opgave med din gruppe og I skal bruge Grouproom, for at snakke/arbejde sammen din grupp og dig. Find en grouprooms hjemsiden
Dominique: Sikke en lausy hjemmeside.
Dominique har let ved denne opgave.2: Opret profil på Grouproom og login og udforsk hjemmesiden.

Dominique: Den er på engelsk. Ved ikke om de skal have efternavn. Den lyser rød, når indholdet ikke er vallidt, og når man skriver sit password. Når man er færdig så lyser den grøn.
Dominique har let ved denne opgave.3: Opret en gruppe for din studiegruppe.
Dominique: Navnet på din første gruppe. Man kan lave flere grupper. Ved ikke hvad upgrade slash scholl code er.
Nu prøver jeg at signe up.
Tjek din spam email. De har sendt email.
Jeg skal aktivere min konto.
Welcome to grouproom.
Det er kendt for brugeren. Projekerne bliver lagt op som tabs. Det minder om Windows.
Jeg kan linke Dropbox og Google Drive.
Opdagelsesturen tager for lang tid.
Dominique har let ved denne opgave.4: Tilføj nogen studiekammerater til denne gruppe.
Mikkel opretter sig på Classroom.
Dominique: Brugeren tror, at det allerede er den gruppe, han har oprettet.
Hvor fanden er der en ny gruppe ?
Membership ? Free profile ? Så må det være den her ?
Hvor fanden laver jeg den nye gruppe ? Public ? Delete ?
Det er ikke til at finde ud af. Brugeren oplever frustration.
Det er et tvetydigt ikon. Opret ny gruppe.
Tilføj nogen studiekammerater: Finder det hurtigt under "members" og "plus". Unødvendig tekst.
Dominique har lidt svært ved denne opgave.5: Upload en fil til gruppen, rediger i denne fil og slet den bagefter.
Dominique: Ohh yaee. Der er billede af, hvad du skal trykke på. Der mangler bar til fil-upload. Det tager lang tid. Brugeren bliver utålmodig og prøver en anden fil.
Filen bliver først uploadet efter, jeg har sagt "refresh".
Nu blev den nye fil uploadet med det samme.
Kan jeg åbne filen ? Man kan ikke redigere filerne online. Man kan kun ændre navnet.
Den bliver slettet uden problemerne.
Dominique har let ved denne opgave.6: Send besked til gruppemedlemmerne.
Dominique: Jeg kan sende beskeder lige her. Hvor er mit gruppe medlem henne. Har du ikke accepteret invitationen ? Der er et ikon der ligner en flamme. Jeg sender en besked nu. Den kan ses af dummy brugeren. Mikkel kan se denne besked.
Dominique har forholdsvis let ved denne opgave.7: Send privatbesked til en af dine studiekammerater.
Dominique: Man kan IKKE sende privatbesked.
Visibility, names, dummy, discussion. Man kan vælge farver på sin event. Blå, rød og gul. Man kan vælge at lave den privat eller gruppe.
Dominique har let ved denne opgave.8: Opret og slet en begivenhed i din gruppes kalender.
Dominique: Calendar, What the fuck. What the fuck. Hvor er delete the task ? Den kan ikke slette begivenhed. Fandt det ved andet forsøg. Det lykkedes at slette den.
Dominique har forholdsvis let ved denne opgave.9: Ekskludér et medlem af gruppen.
Dominique: Det er meget nemt. Der står minus. Man kan fjerne sig selv fra gruppen. Dominique fjerner Mikkel fra gruppen.
Dominique har let ved denne opgave.10: Meld afbud til planlagt begivenhed i din gruppes kalender.
Dominique forsøger at oprette en ny begivenhed for at kunne melde afbud til den. Er i store vanskeligheder. Man kan ikke melde afbud. Man kan kun slette den.
Dominique har svært ved denne opgave.11: Opret en opgave og uddeliger den til et andet medlem.
Dominique fjerner sig selv fra opgaven og giver den til Mikkel.
Dominique har let ved denne opgave.



\end{document}
