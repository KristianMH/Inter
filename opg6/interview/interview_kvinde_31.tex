\documentclass[a4paper,11pt]{report}
\usepackage[T1]{fontenc}
\usepackage[utf8]{inputenc}
\usepackage{lmodern}

\title{Interview}

\begin{document}

\maketitle
 
\begin{enumerate}
\item Fakta\\
  - kvinde 31 år. bruger dagligt sin smartphone i flere sammenhængde(twitter, facebook osv.)
\item Socialklasse "Hvilket erhverv og/eller uddannelsesbaggrund har du?"\\
  - uddannelse: RUC og arbejder til daglig som journalist.
\item Indkøbsfrekvens "Hvor tit handler du ind - og i hvilket tidsrum?"\\
  - 2-3 gange om ugen oftest om eftermiddagen/aften. 
\begin{enumerate}
\item Erfaring	"På hvilket niveau, erfaringsmæssigt, vil du beskrive dig selv som indkøber?"\\
  - Rutineret
\end{enumerate}

\item Tidsfaktor "Hvor lang tid har du til rådighed, når du handler ind?"\\
  - travlt - vil maximalt bruger 20 minutter.
\item Motivation "Hvilken tilgang har du til indkøb - har du eksempelvis en struktureret plan over varer, eller køber du hvad der falder dig ind?"\\
  - jeg bruger ofte en papirs indkøbsseddel hvor jeg skriver de varer og pris på. Skriver oftest pris når der er et specielt tilbud.
\item  Vareplacering "Fortæl mig om en situation, hvor du ikke har kunnet kunne finde en vare i et supermarked."\\
  - Var i Føtex og skulle have grøn pesto. Det kunne jeg ikke se, så jeg spurgte en medarbejder som heller ikke vidste det, 
    dog fik han fat i leder som fandt det for mig. Dette meget lang tid synes jeg
\item Medarb.konfrontation "Hvilke udfordringer forbinder du med at skulle opsøge en medarbejder om en vares placering?"\\
  - Håber at de taler dansk, da jeg synes at der er del udlængige som arbejder i supermarkeder
\item Hukommelse "Hvor mange gange om måneden glemmer du hvad du skal købe i et supermarked? - Kan du fortælle om en specifik situation?"\\
  - Det hænder 1-2 gange om måneden.
\item Teknologisk vane "Hvis du har en smartphone, hvordan bruger du så den i forbindelse med indkøb?"\\
  - Læser nogle gang tilbudsavisen på smartphonen.

\begin{enumerate}
\item Teknologis til-/fravalg "Hvorfor foretrækker du smartphone frem for andre alternativer?"\\
  - Der er lettere at finde tilbudsavisen på smartphonen end at lede i reklamestakken, dog er papir seddlen bedre en indkøbsseddel på smartphone
\end{enumerate}

\item Præventivt varetjek "Hvordan kunne det ændre din indkøbsrutine, hvis du hjemmefra kunne tjekke en vares placering?"\\
  - Dette ville nok kun gøre ved fremmede(nye) supermarkeder, men helt klar hjælpe mig til at blive hurtig færdig.
\item Forventning "Hvad ville du forvente en sådan applikation skulle indeholde?"\\
  - Finde varen, ikke indeholde reklamer.
\item Hurtighed	"Hvor lang tid vil du sige, det højst burde tage at finde en vare ved hjælp af FIVA-appen, hvorfor?"\\
  - 30 sekunder max.
\item Sprogforskelle "Hvilke sprog synes du ville være relevant i forhold til FIVA?"\\
  - Dansk og Engelsk
\item Nødvendighed "Ville du bruge en varelokaliserings-app til din smartphone, hvis sådan en fandtes - hvorfor?"\\
  - Ja, hvis jeg var i et ukendt supermarkedet og skulle finde én vare.
\begin{enumerate}
\item Alternativer "Kan du komme på alternativer, der ville gøre en sådan applikation overflødig?"\\
  - ja Nemlig.com, da folk slet ikke går ud og handler men bare får varene leveret.
\end{enumerate}

\end{enumerate}

\end{document}
