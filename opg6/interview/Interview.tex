\documentclass[12pt]{article}
\usepackage[a4paper, hmargin={2.8cm, 2.8cm}, vmargin={2.5cm, 2.5cm}]{geometry}
\usepackage{eso-pic} % \AddToShipoutPicture

\usepackage[utf8]{inputenc}
\usepackage[T1]{fontenc}
\usepackage{lmodern}
\usepackage[english]{babel}
\usepackage{cite}
\usepackage{amssymb}
\usepackage{amsfonts}
\usepackage{amsmath}
\usepackage{enumerate}
\usepackage{mathrsfs}
\usepackage{fullpage}
\usepackage[linkcolor=red]{hyperref}
\usepackage[final]{graphicx}
\usepackage{color}
\usepackage{listings}
\renewcommand*\lstlistingname{Code Block}
\definecolor{bg}{rgb}{0.95,0.95,0.95}

%caption distinct from normal text
\usepackage[hang,small,bf]{caption}
\usepackage{hyperref}

\hypersetup{
    colorlinks,%
    citecolor=black,%
    filecolor=black,%
    linkcolor=black,%
    urlcolor=black
}

\author{
  \texttt{Gruppe: 3D} \\
  \texttt{Mikkel Enevoldsen} \\[.4cm]
  \texttt{Kristian Høi} \\[.4cm]
  \texttt{Dominique Chancelier} \\[.4cm]
  \texttt{Carsten Jensen} \\[.4cm]
  Instruktor: Jesper Lundsgaard
  \vspace{8cm}
}

\title{
  \vspace{3cm}
  \Huge{Opgave 4} \\[.25cm]
  \large{Målelige krav til brugsvenlighed}
  \vspace{.75cm}
}

\begin{document}

\AddToShipoutPicture*{\put(0,0){\includegraphics*[viewport=0 0 700 600]{includes/ku-farve}}}
\AddToShipoutPicture*{\put(0,602){\includegraphics*[viewport=0 600 700 1600]{includes/ku-farve}}}

%% Change `ku-en` to `nat-en` to use the `Faculty of Science` header
\AddToShipoutPicture*{\put(0,0){\includegraphics*{includes/ku-en}}}

\clearpage\maketitle
\thispagestyle{empty}

\newpage

%\tableofcontents %generate table of content

\thispagestyle{empty}

%\newpage
\pagestyle{plain}
\setcounter{page}{1}
\pagenumbering{arabic}

\section*{PACT}

\textbf{People}
\begin{itemize}
\item Sprogforskelle hvilke sprog synes vil v\ae re passende for FIVA
\item Hukommelse 
\item Generthed (sociale udfordringer ved henvendelse om vareplacering)
\item Socialklasser (indkomstforskelle)
\item Indkøbserfaring
\end{itemize}

\textbf{Activity}
\begin{itemize}
\item Indkøbsfrekvens  
\item Tidspres
\item Formålet veldefineret: Handle ind.
\item Præventivt varetjek. 
\end{itemize}


\textbf{Context}
\begin{itemize}
\item Supermarkeder - indkøb 
\end{itemize}

\textbf{Technology}
\begin{itemize}
\item Smartphone
\item GPS-tilgængelighed
\item Hurtighed
\item Højtlæsning af resultater
\end{itemize}

\newpage

\section*{Tjekliste til interview}
\textbf{Introduktion}\\
"Vi er datalogistuderende fra Københanvs Universitet, som skal designe en applikation omhandlende at gøre det lettere for kunden at finde varer. Vi ønsker at bruge 10 minutter på at få dine tanker omkring en sådan applikation  i et interview."
 
\begin{enumerate}
\item Fakta

\begin{enumerate}
\item Observér: Køn
\item Hvor gammel er du?
\item Hvilken teknologisk erfaring har du? Bruger du smartphone på daglig basis?
\end{enumerate}

\item Socialklasse "Hvilket erhverv og/eller uddannelsesbaggrund har du?"

\item Indkøbsfrekvens "Hvor tit handler du ind - og i hvilket tidsrum?"

\begin{enumerate}
\item Erfaring	"På hvilket niveau, erfaringsmæssigt, vil du beskrive dig selv som indkøber?"
\end{enumerate}

\item Tidsfaktor "Hvor lang tid har du til rådighed, når du handler ind?"
\item Motivation "Hvilken tilgang har du til indkøb - har du eksempelvis en struktureret plan over varer, eller køber du hvad der falder dig ind?"
\item  Vareplacering "Fortæl mig om en situation, hvor du ikke har kunnet kunne finde en vare i et supermarked."

\begin{enumerate}
\item Hjælp "Hvor ofte må du spørge en medarbejder efter hjælp?"
\item Medarbejderfravær "Hvad gør du, når du ikke kan finde en varer og ikke kan komme i kontakt med en medarbejder?"
\end{enumerate}

\item Medarb.konfrontation "Hvilke udfordringer forbinder du med at skulle opsøge en medarbejder om en vares placering?"
\item Hukommelse "Hvor mange gange om måneden glemmer du hvad du skal købe i et supermarked? - Kan du fortælle om en specifik situation?"
\item Teknologisk vane "Hvis du har en smartphone, hvordan bruger du så den i forbindelse med indkøb?"

\begin{enumerate}
\item Teknologisk til-/fravalg "Hvorfor foretrækker du smartphone frem for andre alternativer?"
\end{enumerate}

\item Præventivt varetjek "Hvordan forbereder du dig på en indkøbstur?"

\begin{enumerate}
\item "Hvis vi kigger på de sidste 100 gange du har skullet købe ind, hvor mange gange vil du skyde på at du har ringer og spurgt i forvejen om de har en vare?"
\end{enumerate}

\item Forventning "Hvad ville du forvente en sådan applikation skulle indeholde? Og hvad må den absolut ikke indeholde?"
\item Hurtighed	"Hvor lang tid vil du sige, det højst burde tage at finde en vare ved hjælp af FIVA-appen, hvorfor?"
\item Sprogforskelle "Hvilket sprog synes du FIVA-appen skal være på?"
\item Nødvendighed "Ville du bruge en varelokaliserings-app til din smartphone, hvis sådan en fandtes - hvorfor?"

\begin{enumerate}
\item Alternativer "Kan du komme på alternativer, der ville gøre en sådan applikation overflødig?"
\end{enumerate}

\end{enumerate}
\documentclass[12pt]{article}
\usepackage[a4paper, hmargin={2.8cm, 2.8cm}, vmargin={2.5cm, 2.5cm}]{geometry}
\usepackage{eso-pic} % \AddToShipoutPicture

\usepackage[utf8]{inputenc}
\usepackage[T1]{fontenc}
\usepackage{lmodern}
\usepackage[english]{babel}
\usepackage{cite}
\usepackage{amssymb}
\usepackage{amsfonts}
\usepackage{amsmath}
\usepackage{enumerate}
\usepackage{mathrsfs}
\usepackage{fullpage}
\usepackage[linkcolor=red]{hyperref}
\usepackage[final]{graphicx}
\usepackage{color}
\usepackage{listings}
\renewcommand*\lstlistingname{Code Block}
\definecolor{bg}{rgb}{0.95,0.95,0.95}

%caption distinct from normal text
\usepackage[hang,small,bf]{caption}
\usepackage{hyperref}

\hypersetup{
    colorlinks,%
    citecolor=black,%
    filecolor=black,%
    linkcolor=black,%
    urlcolor=black
}

\author{
  \texttt{Gruppe: 3D} \\
  \texttt{Mikkel Enevoldsen} \\[.4cm]
  \texttt{Kristian Høi} \\[.4cm]
  \texttt{Dominique Chancelier} \\[.4cm]
  \texttt{Carsten Jensen} \\[.4cm]
  Instruktor: Jesper Lundsgaard
  \vspace{8cm}
}

\title{
  \vspace{3cm}
  \Huge{Opgave 4} \\[.25cm]
  \large{Målelige krav til brugsvenlighed}
  \vspace{.75cm}
}

\begin{document}

\AddToShipoutPicture*{\put(0,0){\includegraphics*[viewport=0 0 700 600]{includes/ku-farve}}}
\AddToShipoutPicture*{\put(0,602){\includegraphics*[viewport=0 600 700 1600]{includes/ku-farve}}}

%% Change `ku-en` to `nat-en` to use the `Faculty of Science` header
\AddToShipoutPicture*{\put(0,0){\includegraphics*{includes/ku-en}}}

\clearpage\maketitle
\thispagestyle{empty}

\newpage

%\tableofcontents %generate table of content

\thispagestyle{empty}

%\newpage
\pagestyle{plain}
\setcounter{page}{1}
\pagenumbering{arabic}

\section*{PACT}

\textbf{People}
\begin{itemize}
\item Sprogforskelle hvilke sprog synes vil v\ae re passende for FIVA
\item Hukommelse 
\item Generthed (sociale udfordringer ved henvendelse om vareplacering)
\item Socialklasser (indkomstforskelle)
\item Indkøbserfaring
\end{itemize}

\textbf{Activity}
\begin{itemize}
\item Indkøbsfrekvens  
\item Tidspres
\item Formålet veldefineret: Handle ind.
\item Præventivt varetjek. 
\end{itemize}


\textbf{Context}
\begin{itemize}
\item Supermarkeder - indkøb 
\end{itemize}

\textbf{Technology}
\begin{itemize}
\item Smartphone
\item GPS-tilgængelighed
\item Hurtighed
\item Højtlæsning af resultater
\end{itemize}

\newpage

\section*{Tjekliste til interview}
\textbf{Introduktion}\\
"Vi er datalogistuderende fra Københanvs Universitet, som skal designe en applikation omhandlende at gøre det lettere for kunden at finde varer. Vi ønsker at bruge 10 minutter på at få dine tanker omkring en sådan applikation  i et interview."
 
\begin{enumerate}
\item Fakta

\begin{enumerate}
\item Observér: Køn
\item Hvor gammel er du?
\item Hvilken teknologisk erfaring har du? Bruger du smartphone på daglig basis?
\end{enumerate}

\item Socialklasse "Hvilket erhverv og/eller uddannelsesbaggrund har du?"

\item Indkøbsfrekvens "Hvor tit handler du ind - og i hvilket tidsrum?"

\begin{enumerate}
\item Erfaring	"På hvilket niveau, erfaringsmæssigt, vil du beskrive dig selv som indkøber?"
\end{enumerate}

\item Tidsfaktor "Hvor lang tid har du til rådighed, når du handler ind?"
\item Motivation "Hvilken tilgang har du til indkøb - har du eksempelvis en struktureret plan over varer, eller køber du hvad der falder dig ind?"
\item  Vareplacering "Fortæl mig om en situation, hvor du ikke har kunnet kunne finde en vare i et supermarked."

\begin{enumerate}
\item Hjælp "Hvor ofte må du spørge en medarbejder efter hjælp?"
\item Medarbejderfravær "Hvad gør du, når du ikke kan finde en varer og ikke kan komme i kontakt med en medarbejder?"
\end{enumerate}

\item Medarb.konfrontation "Hvilke udfordringer forbinder du med at skulle opsøge en medarbejder om en vares placering?"
\item Hukommelse "Hvor mange gange om måneden glemmer du hvad du skal købe i et supermarked? - Kan du fortælle om en specifik situation?"
\item Teknologisk vane "Hvis du har en smartphone, hvordan bruger du så den i forbindelse med indkøb?"

\begin{enumerate}
\item Teknologisk til-/fravalg "Hvorfor foretrækker du smartphone frem for andre alternativer?"
\end{enumerate}

\item Præventivt varetjek "Hvordan forbereder du dig på en indkøbstur?"

\begin{enumerate}
\item "Hvis vi kigger på de sidste 100 gange du har skullet købe ind, hvor mange gange vil du skyde på at du har ringer og spurgt i forvejen om de har en vare?"
\end{enumerate}

\item Forventning "Hvad ville du forvente en sådan applikation skulle indeholde? Og hvad må den absolut ikke indeholde?"
\item Hurtighed	"Hvor lang tid vil du sige, det højst burde tage at finde en vare ved hjælp af FIVA-appen, hvorfor?"
\item Sprogforskelle "Hvilket sprog synes du FIVA-appen skal være på?"
\item Nødvendighed "Ville du bruge en varelokaliserings-app til din smartphone, hvis sådan en fandtes - hvorfor?"

\begin{enumerate}
\item Alternativer "Kan du komme på alternativer, der ville gøre en sådan applikation overflødig?"
\end{enumerate}

\end{enumerate}

L\ae rke er en kvinde p\aa 23 \aa r 
Hun er Studerer Media-Videnskab p\aa Suddansk Universitet
Hun handler minimum 3 gange om ugen 
Hun har handlet i 10 \aa r
Det er forskellig, det skal skal nogen gange g\aa hurtigt n\aa r hun handler
Hun har en structureret plan med en indk\o b-liiste hjemmefra, pr\o ver at holder sig til den 
hun har det fint med at sp\o rge medarbejder om hj\ae lp, hun sp\oe rg altid efter hjaelp. 
Hun har ikke s\aa tit brug for hj\ae lp da hun kender varer placering, idet det er en lille Supermarket hun laver sine indk\o b.
Hun hun p\o rger kun om hj\ae lp n\aa r hun handler i en ikke s\aa  bekandt supermarked
F\aar hun ikke hj\ae lp g\aa r i en anden supermarket hvis, hun ingen hj\ae lp f\aa r
Har ikke noget imod at sp\o rger om hj\ae lp
1 ud af 3 gang glemmer hun en varer hun skulle have sk\o bt.
hun bruger sin smartphone til at k\o be ind\ Hendes indk\o b-listen ligger i den. 
har aldrig ringet ind i forvejen f\o r indk\o bs tur.
Finde Vare / vane tilbage til hoved menu, appen skal v\ae re hurtig til s\o ning af vare, Skal komme med forslag, skal ikke have for mange
ting at trykke p\aa, simple. Den skal have mulighed for at lave en indk\o bs listen
Appens sprog skal v\ae r Dansk eller bliver det forvirrende for hende
Hun vil Helt klar bruge den og gl\ae der sig til den udkommer
Hun synes flere medarbejder.

Pascal er en ung mand 19 \aa r 
Han har Ingen smartphone 
Han studerer p\aa Htx 
Han handler 3 til 4 gang om ugen 
Han har 10 \aa r af\ae ring 
Alt dem tid han har bruge for 
Han har en indk\o b list med hver gang
Han Sp\o rger efter hj\ae lp hver gang
Han opgiver hvis Ingen hj\ae lp kommer 
Han synes det kan v\ae r akavet at sp\o rge om hj\ae lp
Han glemmer aldrig da han har en indk\o b list med
Han ringer aldrig for at sp\o rg om varer
han synes appen skal have en Liste, varer i forvejen, man skal kunne m\aa de at kommunikere p\aa, 
Den skal have en liste hvor tingene st\aa r
10 - 20 sek 
Den m\aa godt v\ae r Dansk - Engelsk
Han kunne godt brug appen.


Sidsel Kvinde 41 \aar 
hun er daglige bruger af smartphone
Hun er uddannet Kultur media, erhvers \o konomi
Hun hanlder hver dag,  2 gange om dagen, morgen after b\o rn er afleveret i skole og om eftermiddag
Hun Sp\o rg altid efter hj\ae lp 
Hvis der er ingen hj\ae lp g\aa r hun hen til kassen 
Hun har let ved at sp\o rge om hj\ae lp
Hun sp\o rge kun om hj\ae lp i tilf\ae lde af listen blev glemt.
hun bruger ikke smartphone n\aa r hun k\o ber ind
hun vil v\ae re helt fri for at bruge smartphone til indk\oe b
hun vil Sj\ae lent
Hun er Helt fri for at bruge app
Hj\ae lpe med det sammen \ Dansk - Engelsk
Hun vil kun bruge appen hvis den g\o r indk\o b naermere  eller lettere, og hurtigere.  

\end{document}
