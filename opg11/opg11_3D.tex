\documentclass[12pt]{article}
\usepackage[a4paper, hmargin={2.8cm, 2.8cm}, vmargin={2.5cm, 2.5cm}]{geometry}
\usepackage{eso-pic} % \AddToShipoutPicture

\usepackage[utf8]{inputenc}
\usepackage[T1]{fontenc}
\usepackage{lmodern}
\usepackage[english]{babel}
\usepackage{cite}
\usepackage{amssymb}
\usepackage{amsfonts}
\usepackage{amsmath}
\usepackage{enumerate}
\usepackage{mathrsfs}
\usepackage{fullpage}
\usepackage[linkcolor=red]{hyperref}
\usepackage[final]{graphicx}
\usepackage{color}
\usepackage{listings}
\renewcommand*\lstlistingname{Code Block}
\definecolor{bg}{rgb}{0.95,0.95,0.95}

%caption distinct from normal text
\usepackage[hang,small,bf]{caption}
\usepackage{hyperref}

\hypersetup{
    colorlinks,%
    citecolor=black,%
    filecolor=black,%
    linkcolor=black,%
    urlcolor=black
}

\author{
  \texttt{Gruppe: 3D} \\
  \texttt{Mikkel Enevoldsen} \\[.4cm]
  \texttt{Kristian Høi} \\[.4cm]
  \texttt{Dominique Chancelier} \\[.4cm]
  \texttt{Carsten Jensen} \\[.4cm]
  Instruktor: Jesper Lundsgaard
  \vspace{8cm}
}

\title{
  \vspace{3cm}
  \Huge{Opgave 11} \\[.25cm]
  \large{Diskussion af usabilitytestresultater med kunde}
  \vspace{.75cm}
}

\begin{document}

\AddToShipoutPicture*{\put(0,0){\includegraphics*[viewport=0 0 700 600]{includes/ku-farve}}}
\AddToShipoutPicture*{\put(0,602){\includegraphics*[viewport=0 600 700 1600]{includes/ku-farve}}}

%% Change `ku-en` to `nat-en` to use the `Faculty of Science` header
\AddToShipoutPicture*{\put(0,0){\includegraphics*{includes/ku-en}}}

\clearpage\maketitle
\thispagestyle{empty}

\newpage

\thispagestyle{empty}

\newpage
\pagestyle{plain}
\setcounter{page}{1}
\pagenumbering{arabic}
\section*{Referat af Møde}
\noindent På aktørmødet diskuterede vi med udgangspunkt i Aslaks kommentarer. Fra Grouproom var Andreas Andreas mødt op. Han havde forberedt sig ved at læse vores rapporten og skrevet noter. Undervejs tog han noter når der dukkede gode idéer og kommentarer. \\
Aslak havde fremhævet 3 positive og 3 negative elementer fra samtlige rapporter. 
De positive består af Livechatten, Mulighed for arbejdsstrukturering, general navigation. 
De negative var Søgefeltet, Ikoner og login/sign-up processen. \\

Flere af grupper havde haft positive oplevelser med livechatten hvor de især fremhævede at de var god fordi, den altid var tilgængelige. \\ Mulighed for arbejdsstrukturering dækkede generelt bare kalenderen og task systemet. Dette punkt og general nevaigation blev hurtigt sprunget over, da der ikke var den store interresse for at diskuterer positive elementer.\\

Ved de negative punkter levede diskussionen op igen. Her tog vi udgangspunkt i de tre elementer og slavisk diskuterede dem. \\
Søgebaren, som er en af de punkter vi havde talte om og som vi havde på vores top ti listen. Der kunne ikke blive konsensus om det var tale om en decideret søgefunktion hvor brugeren blev ledt hen til det han/hun søgte på i systemet eller om de filtrede siden og fandt det ord brugeren søgte på. I deres nye version af grouproom har de omdesignet søgebaren og gjort den lettere tilgængelig. \\

Ikoner blev også nævnt her var det stort set en gentagelse af de observationer vi havde med i vores rapport. f.eks. blev der livligt talt om at ikoner var misvisende ved at oprette en gruppe og fjerne gruppe deltagere.\\ I dette punkt blev touren også diskuteret hvor Andreas spurgte hvor mange af os, der ville tage touren, hvor kun få rakte hånden op. De fleste gruppe havde oplevet at den var overflødig og foreslå, at det skulle være en "interaktiv" tour, hvor brugeren blev hjulpet i gang med at lave sin første gruppe.\\

Der var stor enighed om, at login og signin funktionen var et stort problem, og denne blev kritiseret voldsomt under forelæsningen. Flere havde oplevet testdeltagere som havde oplevet problemer med dette. f.eks. at man ikke blev logget ordenligt ud eller at man kan sign-up uden at acceptere "terms and agreement". Især ved disse to eksempler noterede andreas forslag til løsninger, da dette skulle undersøges nærmere.

Andreas præsenterede den nye udgave af Grouproom i et photoshop-dokument. Der var generelt enighed om, at den nye udgave af Grouproom er en forbedring i forhold til den gamle.Især den nye kalender så lovende ud.\\
Andreas spurgte om der var en forbedring, hvor Rolf Molich sørgede for at understrege kraftigt, at den nye udgave af Grouproom kunne ikke vurderes ordentligt, uden at den var blevet testet, da det vil bygge meninger og ikke data.

\end{document}
