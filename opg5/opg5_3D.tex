\documentclass[12pt]{article}
\usepackage[a4paper, hmargin={2.8cm, 2.8cm}, vmargin={2.5cm, 2.5cm}]{geometry}
\usepackage{eso-pic} % \AddToShipoutPicture

\usepackage[utf8]{inputenc}
\usepackage[T1]{fontenc}
\usepackage{lmodern}
\usepackage[english]{babel}
\usepackage{cite}
\usepackage{amssymb}
\usepackage{amsfonts}
\usepackage{amsmath}
\usepackage{enumerate}
\usepackage{mathrsfs}
\usepackage{fullpage}
\usepackage[linkcolor=red]{hyperref}
\usepackage[final]{graphicx}
\usepackage{color}
\usepackage{listings}
\renewcommand*\lstlistingname{Code Block}
\definecolor{bg}{rgb}{0.95,0.95,0.95}

%caption distinct from normal text
\usepackage[hang,small,bf]{caption}
\usepackage{hyperref}

\hypersetup{
    colorlinks,%
    citecolor=black,%
    filecolor=black,%
    linkcolor=black,%
    urlcolor=black
}

\author{
  \texttt{Gruppe: 3D} \\
  \texttt{Mikkel Enevoldsen} \\[.4cm]
  \texttt{Kristian Høi} \\[.4cm]
  \texttt{Dominique Chancelier} \\[.4cm]
  \texttt{Carsten Jensen} \\[.4cm]
  Instruktor: Jesper Lundsgaard
  \vspace{8cm}
}

\title{
  \vspace{3cm}
  \Huge{Opgave 4} \\[.25cm]
  \large{Målelige krav til brugsvenlighed}
  \vspace{.75cm}
}

\begin{document}

\AddToShipoutPicture*{\put(0,0){\includegraphics*[viewport=0 0 700 600]{includes/ku-farve}}}
\AddToShipoutPicture*{\put(0,602){\includegraphics*[viewport=0 600 700 1600]{includes/ku-farve}}}

%% Change `ku-en` to `nat-en` to use the `Faculty of Science` header
\AddToShipoutPicture*{\put(0,0){\includegraphics*{includes/ku-en}}}

\clearpage\maketitle
\thispagestyle{empty}

\newpage

%\tableofcontents %generate table of content

\thispagestyle{empty}

%\newpage
\pagestyle{plain}
\setcounter{page}{1}
\pagenumbering{arabic}

\section*{Fremgangsm\aa de}


\section*{Resultat af indledende PACT-analyse}
Vi indlete projektet med, en omgang brainstorm med udgangs punkt i applictationen FIVA(Finde vare i supermarkedet), af den process fik vi 
en nogen stikord, som vi tog udgangspunkt i til bearbejdning af vores f\o rste interview tjekliste.
 
\textbf{People (Brugere)}
\begin{itemize}
prim\ae re brugergrupper 
\item smartphone brugrere der anvender den til normalbrug

Sekund\ae re m\aa lgrupper
\item Vante smartphone brugere der, \onsker at komme hurtigt igennem indk\o b
\item Vante smartphone brugere der, handler ind til flere dage
\item Smartphone brugere der anvender den til simple brug, som \onsker at komme ubesv\ae ret igennem indk\o b
\item Smartphone brugere, som er pris bevidste og som \onsker sig billige vare.


\item Sprogforskelle hvilke sprog synes vil v\ae re passende for FIVA
\item Hukommelse 
\item Generthed (sociale udfordringer ved henvendelse om vareplacering)
\item Socialklasser (indkomstforskelle)
\item Indkøbserfaring
Brugere har gode erfaring med supermarket indk\o b minimum 1 gang om ugen eller mere.
Brugere har basiske kundskaber i dalig bruge af en smartphone. 
\end{itemize}

\textbf{Activity (Task, Opgaver)}
\begin{itemize}
prim\ae r opgaver 
\item Hj\ae lp med at finde vare i supermarked i forbindelse med indk\o b, s\aa indk\o bsturen kan blive mindre stressende, 
og kortere end mormal 

Sekund\ae r opgaver 
\item Pris sammenligning af vare med n\ae r liggende supermakeder 
\item mulighed for at have indk\o b listen, som inbygget feature i appen.
\item hyppighed - 
\item Indkøbsfrekvens  
\item Tidspres
\item Formålet veldefineret: Handle ind.
\item Præventivt varetjek. 
\end{itemize}


\textbf{Context (Sammenh\ae ng, milj\o )}
\begin{itemize}
Prim\ae r sammenh\ae ng 
\item Brugere anvender FIVA appen i forbindelse med indk\o b i Supermarkeder, og det skal g\aa hurtigere i forhold til normal.
\item By milj\o  ?
\end{itemize}

\textbf{Technology (Teknologi)}
\begin{itemize}
Prim\ae r teknologi
\item Den prim\ae re anvend teknologi er en Smartphone 

Sekund\ae r teknologi
\item GPS-tilgængelighed
\item Hurtighed
\item Højtlæsning af resultater
\end{itemize}

\newpage

\section*{Tjekliste til interview}
\textbf{Introduktion}\\
"Vi er datalogistuderende fra Københanvs Universitet, som skal designe en applikation omhandlende at gøre det lettere for kunden at finde varer. Vi ønsker at bruge 10 minutter på at få dine tanker omkring en sådan applikation  i et interview."
 
\begin{enumerate}
\item Fakta

\begin{enumerate}
\item Observér: Køn
\item Hvor gammel er du?
\item Hvilken teknologisk erfaring har du? Bruger du smartphone på daglig basis?
\end{enumerate}

\item Socialklasse "Hvilket erhverv og/eller uddannelsesbaggrund har du?"

\item Indkøbsfrekvens "Hvor tit handler du ind - og i hvilket tidsrum?"

\begin{enumerate}
\item Erfaring	"På hvilket niveau, erfaringsmæssigt, vil du beskrive dig selv som indkøber?"
\end{enumerate}

\item Tidsfaktor "Hvor lang tid har du til rådighed, når du handler ind?"
\item Motivation "Hvilken tilgang har du til indkøb - har du eksempelvis en struktureret plan over varer, eller køber du hvad der falder dig ind?"
\item  Vareplacering "Fortæl mig om en situation, hvor du ikke har kunnet kunne finde en vare i et supermarked."

\begin{enumerate}
\item Hjælp "Hvor ofte må du spørge en medarbejder efter hjælp?"
\item Medarbejderfravær "Hvad gør du, når du ikke kan finde en varer og ikke kan komme i kontakt med en medarbejder?"
\end{enumerate}

\item Medarb.konfrontation "Hvilke udfordringer forbinder du med at skulle opsøge en medarbejder om en vares placering?"
\item Hukommelse "Hvor mange gange om måneden glemmer du hvad du skal købe i et supermarked? - Kan du fortælle om en specifik situation?"
\item Teknologisk vane "Hvis du har en smartphone, hvordan bruger du så den i forbindelse med indkøb?"

\begin{enumerate}
\item Teknologisk til-/fravalg "Hvorfor foretrækker du smartphone frem for andre alternativer?"
\end{enumerate}

\item Præventivt varetjek "Hvordan forbereder du dig på en indkøbstur?"

\begin{enumerate}
\item "Hvis vi kigger på de sidste 100 gange du har skullet købe ind, hvor mange gange vil du skyde på at du har ringer og spurgt i forvejen om de har en vare?"
\end{enumerate}

\item Forventning "Hvad ville du forvente en sådan applikation skulle indeholde? Og hvad må den absolut ikke indeholde?"
\item Hurtighed	"Hvor lang tid vil du sige, det højst burde tage at finde en vare ved hjælp af FIVA-appen, hvorfor?"
\item Sprogforskelle "Hvilket sprog synes du FIVA-appen skal være på?"
\item Nødvendighed "Ville du bruge en varelokaliserings-app til din smartphone, hvis sådan en fandtes - hvorfor?"

\begin{enumerate}
\item Alternativer "Kan du komme på alternativer, der ville gøre en sådan applikation overflødig?"
\end{enumerate}

\end{enumerate}

\section*{Kerneopgaver (Core tasks)}
Udfaldet af brainstorming og debat mellem gruppe deltager efter interviews b\o d dere efter p\aa nedst\aa ende  
\textbf{Interview}\\

\begin{enumerate}
\item Finde varer - Hylde nummer App'en skal kunne finde varen i butikken. Supermarked taster varens placering ind i deres system.
\item Tjekke om det er billigere hos nabobutikken. App'en skal kunne sammenligne priser på varer i en radius af 300 meter.
\item Søgefunktioner - Skal kunne hjælpe - Autocomplete. Den skal have en søgefunktion. Den skal kunne bidrage med yderligere hjælp, 
hvis kunden bare søger på vin. 
\item Simpel, minimalistisk, let at bruge.
\item Hurtig. Finde resultater indenfor max 10 sekunder. Skal kunne nortere en indkøbsliste.
\item Indkøbsliste. Lav optimal rute i supermarked. Simpel, minimalistisk, let at bruge. 
\end{enumerate}

\textbf{Gruppen}\\
\begin{enumerate}
\item Live gps - Giver rute hen til varen.
\item Nærmeste butik. Finder nærmeste supermarked.
\item Til kassen. Kortest vej til kassen. Der skal være en knap, så man hurtigt kan komme til kassen.Målgrupper:
\end{enumerate}

\begin{enumarate}
Primær målgruppe:
\item Smartphone-bruger der anvender den til simpelt brug, som ønsker at komme ubesværet igennem indkøb.

Sekund\ae r m\aa lgrupper
\item Vante smartphone-brugere, der ønsker at komme hurtigt igennem indkøb.
\item Vante smartphone-brugere, der handler ind til flere dage.
\item Smartphone-bruger der anvender den til simpelt brug, som ønsker at komme ubesværet igennem indkøb.
\item Prisbevidste folk der ønsker billigere varer
\end{enumerate}

\section*{M\aa lgrupper og m\aa lgruppebeskrivelse}

\section*{Personas}

\section*{Scenrier}

\section*{Prototype}

\section*{Erfaringer}

\end{document}
